% !TeX spellcheck = en

%%%%%%%%%%%%%%%%%%%%%%%%%%%%%%%%%%%%%%%%%%%%%%%%%%%%%%%%%%%%%%%%%%%%%%%%%%%%%%%%%%
% SECTION
%%%%%%%%%%%%%%%%%%%%%%%%%%%%%%%%%%%%%%%%%%%%%%%%%%%%%%%%%%%%%%%%%%%%%%%%%%%%%%%%%%	
\section{Grundlagen}

%%%%%%%%%%%%%%%%%%%%%%%%%%%%%%%%%%%%%%%%%%%%%%%%%%%%%%%%%%%%%%%%%%%%%%%%%%%%%%%%%%
% SUBSECTION
%%%%%%%%%%%%%%%%%%%%%%%%%%%%%%%%%%%%%%%%%%%%%%%%%%%%%%%%%%%%%%%%%%%%%%%%%%%%%%%%%%	
\subsection{README}


%%%%%%%%%%%%%%%%%%%%%%%%%%%%%%%%%%%%%%%%%%%%%%%%%%%%%%%%%%%%%%%%%%%%%%%%%%%%%%%%%%
% FRAME
%%%%%%%%%%%%%%%%%%%%%%%%%%%%%%%%%%%%%%%%%%%%%%%%%%%%%%%%%%%%%%%%%%%%%%%%%%%%%%%%%%
\begin{frame}[fragile]
	% [fragile] braucht man, wenn man \verb oder \verbatim auf der Folie verwendet!
	\frametitle{README}
	\begin{itemize}
		\item Das hier ist die Folienvorlage zum Kurs \textbf{Betriebssysteme und Rechnernetze}.
		\item Diese Vorlage soll den Einstieg erleichtern.
		\item Wird unter Linux im Verzeichnis mit der Quelldatei (\texttt{.tex}) das Kommando \texttt{make} eingegeben, wird eine PDF- und eine PS-Datei erzeugt. Beide haben den gleichen Inhalt.
		\item Ein aktuelles \LaTeX\ sollte installiert sein.
		\item Zum Editieren kann ein Texteditor (bspw. Kate) oder eine IDE wie bspw. TeXStudio \cite{TeXStudio} verwendet werden.
		\item Wer unter Windows die Folien machen möchte, dem empfehle ich MiK\TeX\footnote{\texttt{http://www.miktex.org}} \cite{MikTeX} und \TeX nicCenter\footnote{\texttt{http://www.texniccenter.org}} \cite{TexNic}. Hierzu habe ich aber keine Erfahrungswerte.
	\end{itemize}
\end{frame}

\begin{frame}[fragile]
	\frametitle{\LaTeX Beamer}
	\begin{itemize}
		\item Diese Vorlage nutzt die \LaTeX-Klasse \texttt{beamer}.
		\item Eine gute Dokumentation über diese Klasse befindet sich hier:\\
		\url{http://www2.informatik.hu-berlin.de/~mischulz/beamer.html}
		\item Diese Quelle ist auch hilfreich:\\
		\url{http://www.physik.uni-freiburg.de/~tooleh/latex_beamerkurs.pdf}
		\item Google findet sehr viele hilfreiche Links zum Thema \textbf{LaTeX Beamer}.
	\end{itemize}
\end{frame}


%%%%%%%%%%%%%%%%%%%%%%%%%%%%%%%%%%%%%%%%%%%%%%%%%%%%%%%%%%%%%%%%%%%%%%%%%%%%%%%%%%
% SECTION
%%%%%%%%%%%%%%%%%%%%%%%%%%%%%%%%%%%%%%%%%%%%%%%%%%%%%%%%%%%%%%%%%%%%%%%%%%%%%%%%%%	
\section{Textformatierung}

%%%%%%%%%%%%%%%%%%%%%%%%%%%%%%%%%%%%%%%%%%%%%%%%%%%%%%%%%%%%%%%%%%%%%%%%%%%%%%%%%%
% SUBSECTION
%%%%%%%%%%%%%%%%%%%%%%%%%%%%%%%%%%%%%%%%%%%%%%%%%%%%%%%%%%%%%%%%%%%%%%%%%%%%%%%%%%	
\subsection{Schriften und Sonderzeichen}


%%%%%%%%%%%%%%%%%%%%%%%%%%%%%%%%%%%%%%%%%%%%%%%%%%%%%%%%%%%%%%%%%%%%%%%%%%%%%%%%%%
% FRAME
%%%%%%%%%%%%%%%%%%%%%%%%%%%%%%%%%%%%%%%%%%%%%%%%%%%%%%%%%%%%%%%%%%%%%%%%%%%%%%%%%%
\begin{frame}[fragile]
	\frametitle{Schriften und Sonderzeichen}
	\begin{itemize}
		\item Es gibt verschiedene Schriftsätze: \textbf{Bold Face}, \textrm{Roman}, \textit{Italic}, \texttt{Typewriter}, \textsf{Sans Serif}, \textsl{Slanted}, \textsc{Small Caps}.
		\item \textcolor{blue}{Farben} sollte man \textcolor{red}{nicht} zu viel einsetzen.
		\item Ein paar Sonderzeichen: \textbackslash, \$, \&, \euro, \%, \#, \textunderscore, \textasciitilde, \textasciicircum, \textbar, \{, \}
		\item Weitere Sonderzeichen: \copyright, \textregistered, \texttrademark, \S, \P, \pounds, \dag, \ddag, \textbullet
		\item Fortsetzungspunkte macht das Kommando \verb!\dots!. Ergebnis: \dots
	\end{itemize}
\end{frame}

%%%%%%%%%%%%%%%%%%%%%%%%%%%%%%%%%%%%%%%%%%%%%%%%%%%%%%%%%%%%%%%%%%%%%%%%%%%%%%%%%%
% SUBSECTION
%%%%%%%%%%%%%%%%%%%%%%%%%%%%%%%%%%%%%%%%%%%%%%%%%%%%%%%%%%%%%%%%%%%%%%%%%%%%%%%%%%	
\subsection{Schriftgrößen}


%%%%%%%%%%%%%%%%%%%%%%%%%%%%%%%%%%%%%%%%%%%%%%%%%%%%%%%%%%%%%%%%%%%%%%%%%%%%%%%%%%
% FRAME
%%%%%%%%%%%%%%%%%%%%%%%%%%%%%%%%%%%%%%%%%%%%%%%%%%%%%%%%%%%%%%%%%%%%%%%%%%%%%%%%%%
\begin{frame}[fragile]
	\frametitle{Schriftgrößen}
	
	\Huge
	\verb!\Huge!
	\normalsize
	
	\huge
	\verb!\huge!
	\normalsize
	
	\LARGE
	\verb!\LARGE!
	\normalsize
	
	\Large
	\verb!\Large!
	\normalsize
	
	\large
	\verb!\large!
	\normalsize
	
	\normalsize
	\verb!\normalsize!
	\normalsize
	
	\small%%%%%%%%%%%%%%%%%%%%%%%%%%%%%%%%%%%%%%%%%%%%%%%%%%%%%%%%%%%%%%%%%%%%%%%%%%%%%%%%%%
	% FRAME
	%%%%%%%%%%%%%%%%%%%%%%%%%%%%%%%%%%%%%%%%%%%%%%%%%%%%%%%%%%%%%%%%%%%%%%%%%%%%%%%%%%
	\verb!\small!
	\normalsize
	
	\footnotesize
	\verb!\footnotesize!
	\normalsize
	
	\scriptsize
	\verb!\scriptsize!
	\normalsize
	
	\tiny
	\verb!\tiny!
	\normalsize
\end{frame}


%%%%%%%%%%%%%%%%%%%%%%%%%%%%%%%%%%%%%%%%%%%%%%%%%%%%%%%%%%%%%%%%%%%%%%%%%%%%%%%%%%
% SUBSECTION
%%%%%%%%%%%%%%%%%%%%%%%%%%%%%%%%%%%%%%%%%%%%%%%%%%%%%%%%%%%%%%%%%%%%%%%%%%%%%%%%%%	
\subsection{Blöcke}


%%%%%%%%%%%%%%%%%%%%%%%%%%%%%%%%%%%%%%%%%%%%%%%%%%%%%%%%%%%%%%%%%%%%%%%%%%%%%%%%%%
% FRAME
%%%%%%%%%%%%%%%%%%%%%%%%%%%%%%%%%%%%%%%%%%%%%%%%%%%%%%%%%%%%%%%%%%%%%%%%%%%%%%%%%%
\begin{frame}[t]
	% [t] legt fest, dass der Text oben auf der Folie platziert wird.
	\frametitle{Blöcke}
	\begin{itemize}
		\item Es gibt verschiedene Arten von Blöcken:
	\end{itemize}
	\begin{block}{Blocktitel}
		Blocktext
	\end{block}
	
	\begin{exampleblock}{Blocktitel}
		Blocktext
	\end{exampleblock}
	
	\begin{alertblock}{Blocktitel}
		Blocktext
	\end{alertblock}
\end{frame}


%%%%%%%%%%%%%%%%%%%%%%%%%%%%%%%%%%%%%%%%%%%%%%%%%%%%%%%%%%%%%%%%%%%%%%%%%%%%%%%%%%
% SUBSECTION
%%%%%%%%%%%%%%%%%%%%%%%%%%%%%%%%%%%%%%%%%%%%%%%%%%%%%%%%%%%%%%%%%%%%%%%%%%%%%%%%%%	
\subsection{Bilder}


%%%%%%%%%%%%%%%%%%%%%%%%%%%%%%%%%%%%%%%%%%%%%%%%%%%%%%%%%%%%%%%%%%%%%%%%%%%%%%%%%%
% FRAME
%%%%%%%%%%%%%%%%%%%%%%%%%%%%%%%%%%%%%%%%%%%%%%%%%%%%%%%%%%%%%%%%%%%%%%%%%%%%%%%%%%
\begin{frame}
	% [t] legt fest, dass der Text oben auf der Folie platziert wird.
	\frametitle{Bilder}
	\begin{itemize}
		\item Hier ist ein Bild:
	\end{itemize}
	\begin{center}
		\includegraphics[height=4cm]{Pinguin.pdf}
		% Die Dateiendung muss man nicht angeben.
		% Man hätte auch alternativ die Breite mit "width" angeben können.
	\end{center}
	\begin{itemize}
		\item Bilder sollten im Format Encapsulated PostScript (\texttt{.eps}) sein. Dieses Dateiformat kann man mit Gimp und vielen anderen Programmen erzeugen.
	\end{itemize}
\end{frame}


%%%%%%%%%%%%%%%%%%%%%%%%%%%%%%%%%%%%%%%%%%%%%%%%%%%%%%%%%%%%%%%%%%%%%%%%%%%%%%%%%%
% SUBSECTION
%%%%%%%%%%%%%%%%%%%%%%%%%%%%%%%%%%%%%%%%%%%%%%%%%%%%%%%%%%%%%%%%%%%%%%%%%%%%%%%%%%	
\subsection{Tabellen}


%%%%%%%%%%%%%%%%%%%%%%%%%%%%%%%%%%%%%%%%%%%%%%%%%%%%%%%%%%%%%%%%%%%%%%%%%%%%%%%%%%
% FRAME
%%%%%%%%%%%%%%%%%%%%%%%%%%%%%%%%%%%%%%%%%%%%%%%%%%%%%%%%%%%%%%%%%%%%%%%%%%%%%%%%%%
\begin{frame}[fragile]
	\frametitle{Tabellen}
	\begin{itemize}
		\item Es gibt mehrere Umgebungen, um Tabellen zu machen. \texttt{tabular} ist nur eine von vielen.
	\end{itemize}
	
	\begin{center}
		\begin{tabular}{|c|l|c|r|}
			\hline
			\textbf{Zeile} & \textbf{Linksbündig} & \textbf{Zentriert} & \textbf{Rechtsbündig} \\
			\hline\hline
			1 & Zeile 1 & Zeile 1 & Zeile 1 \\
			2 & Zeile 2 & Zeile 2 & Zeile 2 \\
			3 & Zeile 3 & Zeile 3 & Zeile 3 \\
			\hline
		\end{tabular}
	\end{center}
	
	\begin{itemize}
		\item Das geht natürlich auch ohne die Rahmen:
	\end{itemize}
	
	\begin{center}
		\begin{tabular}{clcr}
			\textbf{Zeile} & \textbf{Linksbündig} & \textbf{Zentriert} & \textbf{Rechtsbündig} \\
			1 & Zeile 1 & Zeile 1 & Zeile 1 \\
			2 & Zeile 2 & Zeile 2 & Zeile 2 \\
			3 & Zeile 3 & Zeile 3 & Zeile 3 \\
		\end{tabular}
	\end{center}
\end{frame}


%%%%%%%%%%%%%%%%%%%%%%%%%%%%%%%%%%%%%%%%%%%%%%%%%%%%%%%%%%%%%%%%%%%%%%%%%%%%%%%%%%
% SUBSECTION
%%%%%%%%%%%%%%%%%%%%%%%%%%%%%%%%%%%%%%%%%%%%%%%%%%%%%%%%%%%%%%%%%%%%%%%%%%%%%%%%%%	
\subsection{Mehrspaltige Folien}


%%%%%%%%%%%%%%%%%%%%%%%%%%%%%%%%%%%%%%%%%%%%%%%%%%%%%%%%%%%%%%%%%%%%%%%%%%%%%%%%%%
% FRAME
%%%%%%%%%%%%%%%%%%%%%%%%%%%%%%%%%%%%%%%%%%%%%%%%%%%%%%%%%%%%%%%%%%%%%%%%%%%%%%%%%%
\begin{frame}
	\frametitle{Mehrspaltige Folien}
	\begin{columns}
		\column{.45\textwidth}
		Mehrspaltige Folien können einfach mit \texttt{columns} realisiert werden.
		\column{.45\textwidth}
		\begin{enumerate}
			\item Ein Eintrag
			\item Noch ein Eintrag
		\end{enumerate}
	\end{columns}
\end{frame}

%%%%%%%%%%%%%%%%%%%%%%%%%%%%%%%%%%%%%%%%%%%%%%%%%%%%%%%%%%%%%%%%%%%%%%%%%%%%%%%%%%
% SUBSECTION
%%%%%%%%%%%%%%%%%%%%%%%%%%%%%%%%%%%%%%%%%%%%%%%%%%%%%%%%%%%%%%%%%%%%%%%%%%%%%%%%%%	
\subsection{Zitieren und Quellenangaben}


%%%%%%%%%%%%%%%%%%%%%%%%%%%%%%%%%%%%%%%%%%%%%%%%%%%%%%%%%%%%%%%%%%%%%%%%%%%%%%%%%%
% FRAME
%%%%%%%%%%%%%%%%%%%%%%%%%%%%%%%%%%%%%%%%%%%%%%%%%%%%%%%%%%%%%%%%%%%%%%%%%%%%%%%%%%
\begin{frame}[fragile]
	\frametitle{Ziteren und Quellenangaben}
	
	\begin{itemize}
		\item Quellen können im Text mit \verb|\cite| gefolgt von einem Schlüssel gefolgt werden
		\item Beispielsweise produziert das Kommando\\ \verb|\cite{BibliographyLaTeX}|
		 die Referenz $\rightarrow$ \cite{BibliographyLaTeX}
		 \item Die Quellen werden in einer separaten \texttt{.bib} Datei angelegt
	\end{itemize}
	
\end{frame}